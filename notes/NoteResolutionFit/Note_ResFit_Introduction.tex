% $Id: Note_ResFit_Introduction.tex,v 1.3 2010/05/31 13:10:51 mschrode Exp $


\section{Introduction}

\textit{
Introductory words on jet energy resolution\ldots
\begin{itemize}
\item Gaussian core: calorimeter measurement
\item Low tail: punch-through; $\mu$, $\nu$ from heavy flavour jets
\item High tail: non-linearities of calorimeter
\end{itemize}
}

In this note a method is presented to derive the jet transverse momentum, \pt, resolution from QCD dijet events.
QCD dijet events have a large cross section e.g. in comparison to $\gamma$-jet events and hence provide a higher \pt reach and an early sensitivity also to non-Gaussian tails of the jet response function.

A likelihood function is constructed with probability densities of the \pt imbalance in the dijet event that contain suitable parameterisations of the jet \pt response functions as a function of particle level jet \pt.
A description of selection biases arising from requirements on the measured jet \pt is incorporated.
In case that other event types are included into the likelihood, such as $\gamma+$jets, suitable probability density functions must be defined.

The method is strictly valid in the ideal case of events with only two jets in the final state, that are balanced in \pt at the particle level, and under the assumption that detector response and resolution effects on both jets are not correlated with each other.
In real QCD events, however, the \pt balance of the two jets is affected by the presence of additional jet activity from e.g. gluon radiation and the measurement has to be corrected for this effect.

The studies included in this note are performed on CaloJets although the method is general and may be applied to any type of detector level jet.

