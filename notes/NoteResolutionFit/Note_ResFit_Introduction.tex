% $Id: Note_ResFit_Introduction.tex,v 1.4 2010/06/01 17:24:50 mschrode Exp $


\section{Introduction}

The analysis of jet final states provides an important way to investigate Standard Model and beyond processes and requires a precise knowledge of the jet transverse momentum, \textit{\pt}, resolution.

In this note the jet response, \textit{R}, is defined as the ratio of the measured detector level jet \pt and the underlying particle level jet \pt, \textit{\ptparticle}.
The jet response distribution for a constant \ptparticle follows a Gaussian in good approximation.
Therefore the jet resolution is defined as the standard deviation of the jet response distribution at a given \ptparticle.

\textit{
\begin{itemize}
\item Gaussian core: calorimeter measurement
\item Low tail: punch-through; $\mu$, $\nu$ from heavy flavour jets
\item High tail: non-linearities of calorimeter
\end{itemize}
}

In this note a method is presented to derive the jet transverse momentum, \pt, resolution from QCD dijet events in collider data.
Dijet events have a large cross section e.g. in comparison to $\gamma$-jet events and hence provide a higher \pt reach and an early sensitivity also to non-Gaussian tails of the jet response function.

A likelihood function is constructed with probability densities, \textit{pdf}s, of the \pt imbalance in the dijet event.
The pdfs are a convolution of the particle level jet differential cross section with suitable parameterisations of the jet \pt response functions.
As a consequence, the measured resolution is parametrised as a function of particle level jet \pt.
In addition, a description of selection biases arising from requirements on the measured jet \pt is incorporated.

The method is strictly valid in the ideal case of events with only two jets in the final state, that are balanced in \pt at the particle level, and under the assumption that detector response and resolution effects on both jets are not correlated with each other.
In real QCD events, however, the \pt balance of the two jets is affected by the presence of additional jet activity from e.g. gluon radiation and the measurement has to be corrected for this effect.

In case that other event types are included into the likelihood, such as $\gamma+$jets, suitable probability density functions must be defined.

The studies included in this note are performed on jets that have been reconstructed from calorimeter towers, \textit{\calojets}, although the method is general and may be applied to any type of detector level jet.

