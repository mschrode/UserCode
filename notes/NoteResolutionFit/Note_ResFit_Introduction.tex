% $Id: Note_ResFit_Introduction.tex,v 1.15 2010/10/08 07:52:27 mschrode Exp $


\section{Introduction}

The analysis of jet final states provides an important way to investigate Standard Model and beyond processes and requires a precise knowledge of the jet transverse momentum resolution.
For example, missing transverse energy, \met, due to mismeasured QCD events is an important background to the search for Supersymmetry, \textit{SUSY}, in the inclusive jet channel.
It is foreseen to predict this contribution by smearing QCD seed events with the full jet response function~\cite{bib:ra2}.

For a well calibrated calorimeter, the transverse momentum, \pt, of a jet reconstructed in the detector (\textit{detector level jet}) corresponds on average to the transverse momentum, \ptgen, of the incoming jet clustered from stable particles before interaction with the detector (\textit{particle level jet}).
Here and in the following, \textit{jet \pt response} is defined as the ratio \mbox{$R = \pt/\ptgen$}.
For a fixed \ptgen, the distribution of $R$ follows a Gaussian in good approximation.
Therefore, the \textit{jet \pt resolution} is defined as the standard deviation of the response distribution. 

In this note, a measurement of the response function in a sample of QCD dijet events from $pp$ collisions at \mbox{$\sqrt{s} = 7\tev$} corresponding to an integrated luminosity of $2.2\pbinv$ is presented.
It is performed using an unbinned maximum likelihood fit of the \pt asymmetry in the events.
The likelihood contains a suitable parametrisation of the jet \pt response function as well as an assumption for the particle level jet differential cross section.
The unbinned method has certain advantages in comparison to the binned fit of the histograms of the asymmetry distributions~\cite{bib:asym}.
For example, it enables an unambiguous determination of arbitrary, also asymmetric, response functions and it has a high sensitivity to extremely low populated regions of the response function.
These features are in particular important w.r.t. application in the SUSY search as QCD events with large \met are typically caused by the tiny, $\mathcal{O}(10^{-3})$, non-Gaussian outliers in the response function.
For the studies included in this note, however, only a Gaussian response function is assumed for the time being in order to develop the method and measure the core of the response distribution.
Moreover, biases due to the event selection are explicitly considered in the likelihood.
Therefore a better estimator of \ptgen is provided as argument for the fitted resolution.
Finally, the response can in principle be measured simultaneously also for different event types, such as $\gamma+$jet events and in different $\eta$ regions, thus increasing the covered $\pt$ range and the amount of available events.

The studies included in this note are performed on jets that have been reconstructed from calorimeter towers, although the method is general and may be applied to any type of detector level jet.
Dijets have been chosen due to their relatively large cross section; the likelihood could also be defined for other events such as $\gamma$+jet events, however.

This note is organised as follows:
The general method, including the definition of the dijet likelihood, is described in \qsec{sec:ResFit:Method}.
After listing the event selection in \qsec{sec:ResFit:EvtSel}, the application to Collider data is discussed in \qsec{sec:ResFit:DataDriven}.
Biases due to the selection, showering effects, as well as additional jet activity are investigated and corrected for.
Moreover the performance of the unbinned fit is compared to a binned approach.
Finally, in \qsec{sec:ResFit:Results}, the measured resolution is presented.
