% $Id: Note_ResFit_Main.tex,v 1.6 2010/07/08 14:32:36 mschrode Exp $

% =====  HEADER  ========================================

% ----- Document ----------------------------------------
\documentclass[a4paper]{cmspaper} %USE for CMS machines


% ----- User defined commands ---------------------------
% $ Id: $

% Required pacakges
\usepackage{amsmath}
\usepackage{cancel}
\usepackage{xspace}

\newcommand{\fixme}[1]{\textcolor{red}{{\textbf{FIXME: }\textit{#1}}}}

% Sectioning
\newcommand{\qsec}[1]{Section~\ref{#1}}
\newcommand{\qfig}[1]{Fig.~\ref{#1}}
\newcommand{\qtab}[1]{Table~\ref{#1}}
\newcommand{\qeq}[1]{\eqref{#1}}

% Units
\newcommand{\tev}{\ensuremath{\;\text{Te}\kern-0.06667em\text{V}}\xspace}
\newcommand{\gev}{\ensuremath{\;\text{Ge}\kern-0.06667em\text{V}}\xspace}
\newcommand{\mev}{\ensuremath{\;\text{Me}\kern-0.06667em\text{V}}\xspace}
\newcommand{\km}{\ensuremath{\;\text{km}}\xspace}
\newcommand{\m}{\ensuremath{\;\text{m}}\xspace}
\newcommand{\cm}{\ensuremath{\;\text{cm}}\xspace}
\newcommand{\mm}{\ensuremath{\;\text{mm}}\xspace}
\newcommand{\second}{\ensuremath{\;\text{s}}\xspace}
\newcommand{\tons}{\ensuremath{\;\text{t}}\xspace}
\newcommand{\tesla}{\ensuremath{\;\text{T}}\xspace}
\newcommand{\kelvin}{\ensuremath{\;\text{K}}\xspace}
\newcommand{\pbinv}{\ensuremath{\;\text{pb}^{-1}}\xspace}

% Quantities
\newcommand{\et}{\ensuremath{E_{\text{T}}}\xspace}
\newcommand{\met}{\ensuremath{\slash\mkern-12mu{E}_{\text{T}}}\xspace}
\newcommand{\mht}{\ensuremath{\slash\mkern-12mu{H}_{\text{T}}}\xspace}
\newcommand{\metvec}{\ensuremath{\slash\mkern-12mu{\vec{E}}_{\text{T}}}\xspace}
\newcommand{\pt}{\ensuremath{p_{\text{T}}}\xspace}
\newcommand{\pti}[1]{\ensuremath{p_{\text{T},#1}}\xspace}
\newcommand{\ptvec}{\ensuremath{\vec{p}_{\text{T}}}\xspace}
\newcommand{\ptsub}[1]{\ensuremath{p_{\text{T},#1}}\xspace}
\newcommand{\ptvecsub}[1]{\ensuremath{\vec{p}_{\text{T},#1}}\xspace}
\newcommand{\ptave}{\ensuremath{p^{\text{dijet}}_{\text{T}}}\xspace}
\newcommand{\ptgen}{\ensuremath{p^{\text{gen}}_{\text{T}}}\xspace}
\newcommand{\pthat}{\ensuremath{\hat{p}_{\text{T}}}\xspace}
\newcommand{\pttrue}{\ensuremath{p^{\text{true}}_{\text{T}}}\xspace}
\newcommand{\pttruehigh}[1]{\ensuremath{p^{\text{true,#1}}_{\text{T}}}\xspace}
\newcommand{\ptrel}{\ensuremath{p^{\text{rel}}_{\text{T},3}}\xspace}
\newcommand{\ptcalo}{\ensuremath{p^{\text{calo}}_{\text{T}}}\xspace}
\newcommand{\ptcaloi}[1]{\ensuremath{p^{\text{calo}}_{\text{T},#1}}\xspace}
\newcommand{\ptmin}{\ensuremath{\pt^{\text{min}}}\xspace}
\newcommand{\ptmax}{\ensuremath{\pt^{\text{max}}}\xspace}
\newcommand{\ptreco}{\ensuremath{\pt(\text{reco})}\xspace}
\newcommand{\pp}{\ensuremath{p_{||}}\xspace}
\newcommand{\ppi}[1]{\ensuremath{p_{||,#1}}\xspace}

% Symbols
\newcommand{\dif}[1]{\ensuremath{\text{d}#1}\xspace}
\newcommand{\e}{\,\text{e}}
\newcommand{\nup}[1]{$^{\text{\scriptsize #1}}$}
\newcommand{\dgr}{\ensuremath{\,^{\circ}}}
\newcommand{\mean}[1]{\ensuremath{\langle#1\rangle}}
\newcommand{\gqq}[1]{\ensuremath{\glqq#1\grqq}}
\newcommand{\rarr}{\ensuremath{\rightarrow}\xspace}

% Words and characters
\newcommand{\diagonalsout}[1]{\ensuremath{\cancel{\text{#1}}}}
\newcommand{\genjet}{GenJet\xspace}
\newcommand{\genjets}{GenJets\xspace}
\newcommand{\calojet}{CaloJet\xspace}
\newcommand{\calojets}{CaloJets\xspace}




% ----- Additional packages -----------------------------
\usepackage{subfigure}
\usepackage{booktabs}
\usepackage{multirow}


% ----- pdflatex setup ----------------------------------
\usepackage[pdftex]{color,graphicx}
\usepackage[pdftex]{hyperref}
\hypersetup{unicode=true}
\hypersetup{bookmarks=true}
\hypersetup{pdftitle={Determination of the Jet Energy Resolution in QCD Dijet Events}}
\hypersetup{pdfauthor={Matthias Schr\"oder}}
\hypersetup{colorlinks=true}



% =====  DOCUMENT  ======================================

\begin{document}


% ----- Title page and TOC ------------------------------
\begin{titlepage}
  \date{\today}
  \title{Determination of the Jet Energy Resolution in QCD Dijet
    Events using an Unbinned Maximum Likelihood Method}
  \begin{abstract}
    Analysing jet final states is an important way to investigate Standard Model and beyond processes and requires a precise knowledge of the jet transverse momentum, \pt, resolution.	
    A method is presented to derive the jet \pt resolution from QCD dijet events in collider data.
    It is based on a maximum likelihood fit of the \pt imbalance in the events.	
    Dijet events have a large cross section and hence provide a high \pt reach and an early sensitivity also to non-Gaussian tails of the jet response function.
    However, other event types, such as $\gamma+$jets, can be included into the likelihood.	
  \end{abstract}
\end{titlepage}
\tableofcontents


% ----- Main part from external files -------------------
% $Id: Note_ResFit_Introduction.tex,v 1.3 2010/05/31 13:10:51 mschrode Exp $


\section{Introduction}

\textit{
Introductory words on jet energy resolution\ldots
\begin{itemize}
\item Gaussian core: calorimeter measurement
\item Low tail: punch-through; $\mu$, $\nu$ from heavy flavour jets
\item High tail: non-linearities of calorimeter
\end{itemize}
}

In this note a method is presented to derive the jet transverse momentum, \pt, resolution from QCD dijet events.
QCD dijet events have a large cross section e.g. in comparison to $\gamma$-jet events and hence provide a higher \pt reach and an early sensitivity also to non-Gaussian tails of the jet response function.

A likelihood function is constructed with probability densities of the \pt imbalance in the dijet event that contain suitable parameterisations of the jet \pt response functions as a function of particle level jet \pt.
A description of selection biases arising from requirements on the measured jet \pt is incorporated.
In case that other event types are included into the likelihood, such as $\gamma+$jets, suitable probability density functions must be defined.

The method is strictly valid in the ideal case of events with only two jets in the final state, that are balanced in \pt at the particle level, and under the assumption that detector response and resolution effects on both jets are not correlated with each other.
In real QCD events, however, the \pt balance of the two jets is affected by the presence of additional jet activity from e.g. gluon radiation and the measurement has to be corrected for this effect.

The studies included in this note are performed on CaloJets although the method is general and may be applied to any type of detector level jet.


% $Id: Note_ResFit_Technique.tex,v 1.14 2010/10/08 07:52:28 mschrode Exp $


\section{Description of the Method}\label{sec:ResFit:Method}

\subsection{Definition of the Dijet Probability Density}
In case of ideal dijet events with exactly two jets in the final
state, the transverse momenta of the original partons are balanced due
to momentum conservation.
If hadronisation and jet clustering effects are neglected, also the
\ptgen of the subsequent particle level jets will be equal.
Assuming further that detector response and resolution effects on both jets are not correlated with each other, the probability density (\textit{pdf}) of a dijet event is defined as
\begin{equation}
  \label{eq:ResFit:DijetPdf}
  g_{\mathbf{\xi}}\left(\pti{1},\pti{2}\right) \propto \int\dif{\pttrue}\,f\left(\pttrue\right)
  \cdot r_{\mathbf{\xi}}\left(\pti{1}|\pttrue\right)
  \cdot r_{\mathbf{\xi}}\left(\pti{2}|\pttrue\right) \; ,
\end{equation}
where \pti{i} and \pttrue are the absolute transverse momenta of the
$i$-th jet, \mbox{$i = 1,2$}, at detector level and at particle level, respectively.
Note that the notation \pttrue instead of \ptgen has been chosen in order to emphasise the assumed \pt
balance at particle level, whereas in general \mbox{$\ptgeni{1} \neq \ptgeni{2}$} due to the above
mentioned effects.
Furthermore, $f$ denotes the pdf of \pttrue, \ie the particle level differential jet 
cross section.
In the following, $f$ is taken from the MC
simulation\footnote{The arising systematic uncertainty is evaluated in
  \qsec{sec:ResFit:Systematics}.}
and kept fixed during the maximisation procedure.
$r_{\mathbf{\xi}}(\pti{i}|\pttrue)$ denotes the pdf to measure \pti{i} for the
$i$-th jet given \pttrue.
It is a function that depends on the free parameter set $\mathbf{\xi}$.

It is important to note that $g$ does not depend on \pttrue.
The latter is an integration variable and hence all possible values of \pttrue
contribute to $g$ with the individual weights $f(\pttrue)$.

For a sample of $N$ dijet events, a likelihood is defined as
\begin{equation}
  \label{eq:ResFit:Likelihood}
  \mathcal{L}\left(\mathbf{\xi}\right) = \prod^{N}_{k=1} g_{\mathbf{\xi},k}\left(\pti{1},\pti{2}\right).
\end{equation}
The values of $\mathbf{\xi}$ which maximise $\mathcal{L}$ correspond
to the response $r_{\mathbf{\xi}}$ that describes the data best.
The pdf of the jet \pt response \mbox{$R = \pt / \pttrue$} is obtained by parameter transformation
\begin{equation*}
  r_{\mathbf{\xi}}\left(R|\pttrue\right) =
  r_{\mathbf{\xi}}\left(\pt\left(R\right)|\pttrue\right)\cdot\left|\frac{\dif{\pt}}{\dif{R}}\right|.
\end{equation*}


\subsection{Case of a Gaussian Response}

No assumption about the functional form of $r_{\mathbf{\xi}}$ has been
made so far.
It can be parametrised with any function and hence also
describe unambiguously extreme, non-Gaussian outliers of the response which are in
in particular responsible for QCD events with large \met.
However in the following, a Gaussian function is assumed for the time
being in order to develop the method and measure the main bulk of the
response.
The Gaussian have a fixed mean value of one, \ie the
correct jet energy scale is assumed, and be dependent on the free width
parameter \mbox{$\mathcal{\xi} = \sigma$}.
In that case a different choice of coordinates than $\pti{1}$ and
$\pti{2}$ is preferential.
With
\begin{align}
  \label{eq:ResFit:TransformedCoordinates}
  \begin{split}
    \ptave     &  = \frac{1}{2}\left(\pti{1}+\pti{2}\right) \\
    \Delta\pt  &  = \frac{1}{2}\left(\pti{1}-\pti{2}\right)
  \end{split}
\end{align} 
the dijet pdf~\qeq{eq:ResFit:DijetPdf} becomes
\begin{equation}
  \label{eq:ResFit:DijetPdfTransformed}
   g_{\sigma'}\left(\ptave,\Delta\pt\right) \propto
   \e^{-\frac{1}{2}\left(\frac{\Delta\pt}{\sigma'}\right)^{2}}\cdot
   \int\dif{\pttrue}\;f\left(\pttrue\right)\cdot
   \e^{-\frac{1}{2}\left(\frac{\ptave - \pttrue}{\sigma'}\right)^{2}}
   \; ,
\end{equation}
where \mbox{$\sigma' = \sigma/\sqrt{2}$}.
It is evident from~\qeq{eq:ResFit:DijetPdfTransformed} that the dijet
pdf is in fact equivalent to a pdf of the \pt imbalance, \ie the dijet
asymmetry, with an additional assumption for the particle level jet
\pt spectrum.
This will be further discussed and exploited in \qsec{sec:ResFit:DataDriven}.

% $Id: $


\section{Toy Monte Carlo Simulation}


\subsection{Generated sample}


\subsection{Measurement of the resolution with truth information based event selection}


\subsection{Measurement of the resolution with data driven event selection}

% $Id: Note_ResFit_QCDMC.tex,v 1.1 2010/05/30 19:43:08 mschrode Exp $


\section{Study with a QCD Monte Carlo Simulation}\label{sec:ResFit:QCDMC}

The method is also tested on a sample of simulated QCD multijet events in $pp$ collisions at 7\tev center-of-mass energy.
They have been generated with PYTHIA and processed through the full CMS detector simulation application based on GEANT4\footnote{The used dataset is \texttt{/QCDFlat\_Pt15to3000/Spring10-START3X\_V26\_S09-v1/GEN-SIM-RECO}}.
The events are weighted corresponding to an integrated luminosity of 50\pbinv.


\subsection{Event selection}\label{sec:ResFit:QCDMC:EvtSel}

Jets are reconstructed from calorimeter towers using the anti-$k_{T}$ jet clustering algorithm~\cite{bib:akj} with parameter size $d=0.5$.
Jet energy corrections~\cite{bib:cmspas:jec} are applied to remove the $\eta$ and \pt dependence of the jet energy scale.
Dijet events are selected in the following way:
\begin{enumerate}
\item The third jet is required to have small \pt compared to the leading two jets by imposing \mbox{$\ptrel < x$}, with \mbox{$\ptrel = \frac{2\pti{3}}{\pti{1} + \pti{2}}$}, in order to ensure a dijet topology.
  The value of $x$ is varied in course of the analysis as discussed in Section~\ref{sec:ResFit:QCDMC:AddJetAct}.
\item Both leading jets are required to be in the same pseudorapidity bin, which are defined in Table~\ref{tab:ResFit:QCDMC:EtaBinning}.
  This accounts for the dependence of the jet \pt response on $\eta$ due to
  \begin{itemize}
  \item different calorimeter geometries and material budgets in front of the calorimeters in different pseudorapidity regions;
  \item the fact that the intrinsic calorimeter resolution is energy rather than \pt dependent.
  \end{itemize}
\item In order to reject jets clustered from noise in the hadronic calorimeter component, the fraction $f_{\text{em}}$ of energy deposited in the electromagnetic component is required to be \mbox{$f_{\text{em}} > 0.01$} for both leading jets~\cite{bib:cmspas:jetid}.
\end{enumerate}


\subsection{Gaussian resolution from Monte Carlo truth information}\label{sec:ResFit:QCDMC:MCTruthReso}

The jet \pt resolution is determined from Monte Carlo truth information, first, as a reference to compare the measured resolution to, and second, as an input $r_{0}$ in~\eqref{eq:ResFit:Method:ModifiedSpectrum} to incorporate the the selection bias into the dijet likelihood (comp. Section~\ref{sec:ResFit:Method:Biases}).
\begin{figure}[ht]
  \begin{center}
     \includegraphics[width=0.45\textwidth]{figures/resFit_QCD_MCTruthResolution}
   \end{center}
   \caption{Relative Gaussian jet \pt resolution $\sigma/\pt$ derived from Monte Carlo truth as a function of \pt in different $\eta$ bins.}
   \label{fig:ResFit:ToyMC:Sample:Spectrum}
\end{figure}
The determination follows closely~\cite{}:
In each event, the two jets with highest particle level jet \pt, \ptparticle, are selected and the response distributions \mbox{$\ptmeas / \ptparticle$} are recorded in different bins of \ptparticle and $\eta$ (the $\eta$ binning is listed in Table~\ref{tab:ResFit:QCDMC:EtaBinning}).
The central part of each distribution --- defined by the interval of 1.5 standard deviations around the mean --- is fitted with a Gaussian.
In each $\eta$ bin, the fitted Gaussian widths $\sigma$ are interpolated for different \pt by the usual parametrisation function~\eqref{eq:ResFit:ToyMC:Sigma}; the fitted parameter values are listed in Table~\ref{tab:ResFit:QCDMC:MCTruthReso}.
\begin{table}[ht]
  \caption{Parameters of the MC truth resolution $\sigma$.}
  \begin{center}
    \begin{tabular}[h]{cccccc}
      \toprule
      & $|\eta_{\text{min}}|$ & $|\eta_{\text{max}}|$ & $\xi_{0}\,(\text{Ge}\kern-0.06667em\text{V})$ & $\xi_{1}\,(\sqrt{\text{Ge}\kern-0.06667em\text{V}})$ & $\xi_{2}$ \\
      \midrule
      $0$ & $0$ & $1.2$ & $1.9\pm0.2$ & $1.205\pm0.006$  & $0.0342\pm0.0007$ \\
      $1$ & $1.2$ & $2.6$ & $2.51\pm0.09$ & $0.968\pm0.009$  & $0.0483\pm0.0009$ \\
      $2$ & $2.6$ & $3.2$ & $1.7\pm0.18$ & $0.76\pm0.02$  & $0.035\pm0.004$ \\
      $3$ & $3.2$ & $4.7$ & $2.2\pm0.1$ & $0.2\pm0.1$  & $0.099\pm0.003$ \\
      \bottomrule
    \end{tabular}
  \end{center}
  \label{tab:ResFit:QCDMC:MCTruthReso}
\end{table}




\subsection{Effects from additional jet activity}\label{sec:ResFit:QCDMC:AddJetAct}

% Resolutions are derived from Gaussian fits to the simulated distributions~\eqref{eq:qcd:resolMaxlike:toyMCRes}.
% Figure.~\ref{fig:qcd:resolMaxlike:qcd:ptDependentSigma} illustrates on the lack of closure due to the presence of a third jet. 

% \begin{figure}[ht]
%   \begin{center}
%     \subfigure[]{
%       \includegraphics[width=0.45\textwidth]{figures/figures_QCD_resol_maxlike/resFit_PtDependentSigma}
%     } \subfigure[]{
%       \includegraphics[width=0.45\textwidth]{figures/figures_QCD_resol_maxlike/resFit_QCD_3rdJetInfluence}
%     }
%   \end{center}
%   \caption{(a) Gaussian jet \pt resolution in realistic jet
% events generated with PYTHIA and processed through the CMS full simulation
% application based on GEANT.
% $\sigma$ from fitted parameter values (red line) is compared to the MC truth prediction (blue line).
% (This is the result of a study on the \texttt{/QCDFlat\_Pt15to3000/Summer09-MC\_31X\_V9\_7TeV-v1/GEN-SIM-RECO} dataset. It will be updated to the Spring10 simulation but the differences are expected to be negligible.)
% (b) Illustration using a toy simulation of the effect of three jet events on the fitted 
% $\sigma$.}
%       \label{fig:qcd:resolMaxlike:qcd:ptDependentSigma}
% \end{figure}


% \begin{figure}[ht]
%   \centering
%   \includegraphics[width=0.45\textwidth]{figures/resFit_PtDependentSigma}
%   \caption{Width $\sigma$ of a Gaussian resolution in QCD dijet events. Shown is $\sigma$ from fitted parameter values (red line) in comparison to the truth from the MC simulation (blue line).}
%   \label{fig:resFit:qcd:ptDependentSigma}
% \end{figure}

% The determination of a Gaussian resolution~\eqref{eq:resFit:toyMCRes} with a \pt dependent width $\sigma$ as in~\eqref{eq:resFit:toyMCSigma} has been performed as for the ideal sample above.
% The results are not satisfying (comp. Fig.~\ref{fig:resFit:qcd:ptDependentSigma}).

% Various test have been performed and the method appears to work fine for a moderately falling \pt spectrum.
% In case of a realistic QCD spectrum (comp. Fig.~\ref{fig:resFit:qcd:dijetspectrum:subA}), however, the
% determined resolution does not describe the true resolution at large \pt anymore.
% The failure of the method is assumed to be due to the non-ideal topology of the selected dijet events, namely the presence of a third jet.
% These effects and an extension of the presented method are under study.
% Meanwhile a modified strategy to determine the resolution is investigated and presented in the following sections.

% $Id: Note_ResFit_QCDMC_Extrapolation.tex,v 1.1 2010/05/30 19:43:08 mschrode Exp $


\subsection{Extrapolation to the Two Jet Final State Topology}

\textit{Elaborate on idea\ldots}


\subsubsection{Measurement of the Gaussian resolution}
\begin{table}[ht]
  \caption{$\eta$ binning}
  \centering
  \begin{tabular}{cl|ccccccccc}
    \toprule
    \multicolumn{2}{c}{$|\eta|$ bin} & \multicolumn{9}{c}{\pt bin edges $(\text{Ge}\kern-0.06667em\text{V})$} \\
    \midrule
    \multirow{2}{*}{$0$} & \multirow{2}{*}{$(0 - 1.2)$} & 80 & 100 & 120 & 140 & 170 & 200 & 250 & 300 & 350 \\
    && 400 & 500 & 600 & 800 & 1000 \\
    $1$ & $(1.2 - 2.6)$ & 60 & 80 & 100 & 120 & 150 & 200 & 300 &  500 & 700 \\
    $2$ & $(2.6 - 3.2)$ & 80 & 100 & 150 &&&&&&\\
    \bottomrule
  \end{tabular}
  \label{tab:ResFit:QCDMC:PtBinning}
\end{table}


\subsubsection{Measurement of the resolution}

%% $Id: Note_ResFit_Todo.tex,v 1.1 2010/06/01 17:24:13 mschrode Exp $


\section{Todo\ldots}

\subsection{Analysis}
\begin{itemize}
\item Extrapolation of (correlated!) Crystal Ball parameters ok?
\item Difference $\pt^{\text{parton}} \leftrightarrow \pt^{\text{particleJet}}$
\end{itemize}


\subsection{Note}

\subsubsection{Content}
\begin{itemize}
\item Advantage of an unbinned fit: works for less events
  \begin{itemize}
  \item Higher \pt reach
  \item Sensitivity to tails (paragraph on why tails are important)
  \end{itemize}
\item Advantage of a \pt dependent $\sigma$
  \begin{itemize}
  \item Fitted $\sigma$ also in spectrum (resolutin bias description)
  \end{itemize}
\item Motivate parametrisation of resolution in \pt
\end{itemize}

\subsubsection{Style}
\begin{itemize}
\item Define consequently all quantities, in particular: $\pt^{\text{particle}}$ is particle jet \pt
\end{itemize}

\appendix
\section{Fit results for all \pt and $\eta$ bins}\label{sec:ResFit:App:AllResults}
\subsection{Gaussian response function}\label{sec:ResFit:App:AllResults:Gauss}

% ----- Gauss Eta0 Spectra -----
\begin{figure}[ht]
  \centering
  \begin{tabular}{ccc}
    \includegraphics[width=0.3\textwidth]{figures/ResFit_Spring10QCDFlat_Gauss_Eta0_Spectrum_PtBin1} &
    \includegraphics[width=0.3\textwidth]{figures/ResFit_Spring10QCDFlat_Gauss_Eta0_Spectrum_PtBin2} &
    \includegraphics[width=0.3\textwidth]{figures/ResFit_Spring10QCDFlat_Gauss_Eta0_Spectrum_PtBin3} \\

    \includegraphics[width=0.3\textwidth]{figures/ResFit_Spring10QCDFlat_Gauss_Eta0_Spectrum_PtBin4} &
    \includegraphics[width=0.3\textwidth]{figures/ResFit_Spring10QCDFlat_Gauss_Eta0_Spectrum_PtBin5} &
    \includegraphics[width=0.3\textwidth]{figures/ResFit_Spring10QCDFlat_Gauss_Eta0_Spectrum_PtBin6} \\

    \includegraphics[width=0.3\textwidth]{figures/ResFit_Spring10QCDFlat_Gauss_Eta0_Spectrum_PtBin7} &
    \includegraphics[width=0.3\textwidth]{figures/ResFit_Spring10QCDFlat_Gauss_Eta0_Spectrum_PtBin8} &
    \includegraphics[width=0.3\textwidth]{figures/ResFit_Spring10QCDFlat_Gauss_Eta0_Spectrum_PtBin9} \\

    \includegraphics[width=0.3\textwidth]{figures/ResFit_Spring10QCDFlat_Gauss_Eta0_Spectrum_PtBin10} &
    \includegraphics[width=0.3\textwidth]{figures/ResFit_Spring10QCDFlat_Gauss_Eta0_Spectrum_PtBin11} &
    \includegraphics[width=0.3\textwidth]{figures/ResFit_Spring10QCDFlat_Gauss_Eta0_Spectrum_PtBin12} \\
  \end{tabular}
\caption{The parameterisation of the realistic particle jet \pt spectrum as used in the dijet likelihood (solid line) in comparison to the prediction from Monte Carlo truth (full circles) in different \pt bins for \mbox{$|\eta|<1.2$}. Migration effects are modeled assuming a Gaussian response function.}
\label{fig:ResFit:App:Gauss:Spectrum}
\end{figure}


% ----- Gauss Eta0 Extrapolations -----
\begin{figure}[ht]
  \centering
  \begin{tabular}{ccc}
    \includegraphics[width=0.3\textwidth]{figures/ResFit_Spring10QCDFlat_Gauss_Eta0_ExtrapolatedPar0_PtBin1} &
    \includegraphics[width=0.3\textwidth]{figures/ResFit_Spring10QCDFlat_Gauss_Eta0_ExtrapolatedPar0_PtBin2} &
    \includegraphics[width=0.3\textwidth]{figures/ResFit_Spring10QCDFlat_Gauss_Eta0_ExtrapolatedPar0_PtBin3} \\

    \includegraphics[width=0.3\textwidth]{figures/ResFit_Spring10QCDFlat_Gauss_Eta0_ExtrapolatedPar0_PtBin4} &
    \includegraphics[width=0.3\textwidth]{figures/ResFit_Spring10QCDFlat_Gauss_Eta0_ExtrapolatedPar0_PtBin5} &
    \includegraphics[width=0.3\textwidth]{figures/ResFit_Spring10QCDFlat_Gauss_Eta0_ExtrapolatedPar0_PtBin6} \\

    \includegraphics[width=0.3\textwidth]{figures/ResFit_Spring10QCDFlat_Gauss_Eta0_ExtrapolatedPar0_PtBin7} &
    \includegraphics[width=0.3\textwidth]{figures/ResFit_Spring10QCDFlat_Gauss_Eta0_ExtrapolatedPar0_PtBin8} &
    \includegraphics[width=0.3\textwidth]{figures/ResFit_Spring10QCDFlat_Gauss_Eta0_ExtrapolatedPar0_PtBin9} \\

    \includegraphics[width=0.3\textwidth]{figures/ResFit_Spring10QCDFlat_Gauss_Eta0_ExtrapolatedPar0_PtBin10} &
    \includegraphics[width=0.3\textwidth]{figures/ResFit_Spring10QCDFlat_Gauss_Eta0_ExtrapolatedPar0_PtBin11} &
    \includegraphics[width=0.3\textwidth]{figures/ResFit_Spring10QCDFlat_Gauss_Eta0_ExtrapolatedPar0_PtBin12} \\
  \end{tabular}
\caption{Resolutions $\bar{\sigma}/\pt$ from Gaussian fits for different limits on \ptrel in different \pt bins for \mbox{$|\eta|<1.2$}.
  The solid line is a linear fit to extrapolate $\bar{\sigma}/\pt$ to the ideal case of only two jets in the
  final state.}
\label{fig:ResFit:App:Gauss:Extrapolation}
\end{figure}


% ----- Gauss Eta0 MC truth closure -----
\begin{figure}[ht]
  \centering
  \begin{tabular}{ccc}
    \includegraphics[width=0.3\textwidth]{figures/ResFit_Spring10QCDFlat_Gauss_Eta0_MCClosure_PtBin1} &
    \includegraphics[width=0.3\textwidth]{figures/ResFit_Spring10QCDFlat_Gauss_Eta0_MCClosure_PtBin2} &
    \includegraphics[width=0.3\textwidth]{figures/ResFit_Spring10QCDFlat_Gauss_Eta0_MCClosure_PtBin3} \\

    \includegraphics[width=0.3\textwidth]{figures/ResFit_Spring10QCDFlat_Gauss_Eta0_MCClosure_PtBin4} &
    \includegraphics[width=0.3\textwidth]{figures/ResFit_Spring10QCDFlat_Gauss_Eta0_MCClosure_PtBin5} &
    \includegraphics[width=0.3\textwidth]{figures/ResFit_Spring10QCDFlat_Gauss_Eta0_MCClosure_PtBin6} \\

    \includegraphics[width=0.3\textwidth]{figures/ResFit_Spring10QCDFlat_Gauss_Eta0_MCClosure_PtBin7} &
    \includegraphics[width=0.3\textwidth]{figures/ResFit_Spring10QCDFlat_Gauss_Eta0_MCClosure_PtBin8} &
    \includegraphics[width=0.3\textwidth]{figures/ResFit_Spring10QCDFlat_Gauss_Eta0_MCClosure_PtBin9} \\

    \includegraphics[width=0.3\textwidth]{figures/ResFit_Spring10QCDFlat_Gauss_Eta0_MCClosure_PtBin10} &
    \includegraphics[width=0.3\textwidth]{figures/ResFit_Spring10QCDFlat_Gauss_Eta0_MCClosure_PtBin11} &
    \includegraphics[width=0.3\textwidth]{figures/ResFit_Spring10QCDFlat_Gauss_Eta0_MCClosure_PtBin12} \\
  \end{tabular}
\caption{Closure \mbox{$|\eta|<1.2$}.}
\label{fig:ResFit:App:Gauss:MCClosure}
\end{figure}

\clearpage


\subsection{Crystal Ball response function}\label{sec:ResFit:App:AllResults:CrystalBall}

% ----- Crystal Ball Eta0 Spectra -----
\begin{figure}[ht]
  \centering
  \begin{tabular}{ccc}
    \includegraphics[width=0.3\textwidth]{figures/ResFit_Spring10QCDFlat_CB_Eta0_Spectrum_PtBin0} &
    \includegraphics[width=0.3\textwidth]{figures/ResFit_Spring10QCDFlat_CB_Eta0_Spectrum_PtBin1} &
    \includegraphics[width=0.3\textwidth]{figures/ResFit_Spring10QCDFlat_CB_Eta0_Spectrum_PtBin2} \\

    \includegraphics[width=0.3\textwidth]{figures/ResFit_Spring10QCDFlat_CB_Eta0_Spectrum_PtBin3} &
    \includegraphics[width=0.3\textwidth]{figures/ResFit_Spring10QCDFlat_CB_Eta0_Spectrum_PtBin4} &
    \includegraphics[width=0.3\textwidth]{figures/ResFit_Spring10QCDFlat_CB_Eta0_Spectrum_PtBin5} \\

    \includegraphics[width=0.3\textwidth]{figures/ResFit_Spring10QCDFlat_CB_Eta0_Spectrum_PtBin6} &
    \includegraphics[width=0.3\textwidth]{figures/ResFit_Spring10QCDFlat_CB_Eta0_Spectrum_PtBin7} &
    \includegraphics[width=0.3\textwidth]{figures/ResFit_Spring10QCDFlat_CB_Eta0_Spectrum_PtBin8} \\

    \includegraphics[width=0.3\textwidth]{figures/ResFit_Spring10QCDFlat_CB_Eta0_Spectrum_PtBin9} &
    \includegraphics[width=0.3\textwidth]{figures/ResFit_Spring10QCDFlat_CB_Eta0_Spectrum_PtBin10} & \\
  \end{tabular}
\caption{The parameterisation of the realistic particle jet \pt spectrum as used in the dijet likelihood (solid line) in comparison to the prediction from Monte Carlo truth (full circles) in different \pt bins for \mbox{$|\eta|<1.2$}. Migration effects are modeled assuming a Crystal Ball response function.}
\label{fig:ResFit:App:CB:Spectrum}
\end{figure}


% ----- Crystal Ball Eta0 extrapolations -----
\begin{figure}[ht]
  \centering
  \begin{tabular}{ccc}
    \includegraphics[width=0.3\textwidth]{figures/ResFit_Spring10QCDFlat_CB_Eta0_ExtrapolatedPar0_PtBin0} &
    \includegraphics[width=0.3\textwidth]{figures/ResFit_Spring10QCDFlat_CB_Eta0_ExtrapolatedPar0_PtBin1} &
    \includegraphics[width=0.3\textwidth]{figures/ResFit_Spring10QCDFlat_CB_Eta0_ExtrapolatedPar0_PtBin2} \\

    \includegraphics[width=0.3\textwidth]{figures/ResFit_Spring10QCDFlat_CB_Eta0_ExtrapolatedPar0_PtBin3} &
    \includegraphics[width=0.3\textwidth]{figures/ResFit_Spring10QCDFlat_CB_Eta0_ExtrapolatedPar0_PtBin4} &
    \includegraphics[width=0.3\textwidth]{figures/ResFit_Spring10QCDFlat_CB_Eta0_ExtrapolatedPar0_PtBin5} \\

    \includegraphics[width=0.3\textwidth]{figures/ResFit_Spring10QCDFlat_CB_Eta0_ExtrapolatedPar0_PtBin6} &
    \includegraphics[width=0.3\textwidth]{figures/ResFit_Spring10QCDFlat_CB_Eta0_ExtrapolatedPar0_PtBin7} &
    \includegraphics[width=0.3\textwidth]{figures/ResFit_Spring10QCDFlat_CB_Eta0_ExtrapolatedPar0_PtBin8} \\

    \includegraphics[width=0.3\textwidth]{figures/ResFit_Spring10QCDFlat_CB_Eta0_ExtrapolatedPar0_PtBin9} &
    \includegraphics[width=0.3\textwidth]{figures/ResFit_Spring10QCDFlat_CB_Eta0_ExtrapolatedPar0_PtBin10} & \\
  \end{tabular}
\caption{}
\label{fig:ResFit:App:CB:ExtrapolatedPar0}
\end{figure}

\begin{figure}[ht]
  \centering
  \begin{tabular}{ccc}
    \includegraphics[width=0.3\textwidth]{figures/ResFit_Spring10QCDFlat_CB_Eta0_ExtrapolatedPar1_PtBin0} &
    \includegraphics[width=0.3\textwidth]{figures/ResFit_Spring10QCDFlat_CB_Eta0_ExtrapolatedPar1_PtBin1} &
    \includegraphics[width=0.3\textwidth]{figures/ResFit_Spring10QCDFlat_CB_Eta0_ExtrapolatedPar1_PtBin2} \\

    \includegraphics[width=0.3\textwidth]{figures/ResFit_Spring10QCDFlat_CB_Eta0_ExtrapolatedPar1_PtBin3} &
    \includegraphics[width=0.3\textwidth]{figures/ResFit_Spring10QCDFlat_CB_Eta0_ExtrapolatedPar1_PtBin4} &
    \includegraphics[width=0.3\textwidth]{figures/ResFit_Spring10QCDFlat_CB_Eta0_ExtrapolatedPar1_PtBin5} \\

    \includegraphics[width=0.3\textwidth]{figures/ResFit_Spring10QCDFlat_CB_Eta0_ExtrapolatedPar1_PtBin6} &
    \includegraphics[width=0.3\textwidth]{figures/ResFit_Spring10QCDFlat_CB_Eta0_ExtrapolatedPar1_PtBin7} &
    \includegraphics[width=0.3\textwidth]{figures/ResFit_Spring10QCDFlat_CB_Eta0_ExtrapolatedPar1_PtBin8} \\

    \includegraphics[width=0.3\textwidth]{figures/ResFit_Spring10QCDFlat_CB_Eta0_ExtrapolatedPar1_PtBin9} &
    \includegraphics[width=0.3\textwidth]{figures/ResFit_Spring10QCDFlat_CB_Eta0_ExtrapolatedPar1_PtBin10} & \\
  \end{tabular}
\caption{}
\label{fig:ResFit:App:CB:ExtrapolatedPar1}
\end{figure}

\begin{figure}[ht]
  \centering
  \begin{tabular}{ccc}
    \includegraphics[width=0.3\textwidth]{figures/ResFit_Spring10QCDFlat_CB_Eta0_ExtrapolatedPar2_PtBin0} &
    \includegraphics[width=0.3\textwidth]{figures/ResFit_Spring10QCDFlat_CB_Eta0_ExtrapolatedPar2_PtBin1} &
    \includegraphics[width=0.3\textwidth]{figures/ResFit_Spring10QCDFlat_CB_Eta0_ExtrapolatedPar2_PtBin2} \\

    \includegraphics[width=0.3\textwidth]{figures/ResFit_Spring10QCDFlat_CB_Eta0_ExtrapolatedPar2_PtBin3} &
    \includegraphics[width=0.3\textwidth]{figures/ResFit_Spring10QCDFlat_CB_Eta0_ExtrapolatedPar2_PtBin4} &
    \includegraphics[width=0.3\textwidth]{figures/ResFit_Spring10QCDFlat_CB_Eta0_ExtrapolatedPar2_PtBin5} \\

    \includegraphics[width=0.3\textwidth]{figures/ResFit_Spring10QCDFlat_CB_Eta0_ExtrapolatedPar2_PtBin6} &
    \includegraphics[width=0.3\textwidth]{figures/ResFit_Spring10QCDFlat_CB_Eta0_ExtrapolatedPar2_PtBin7} &
    \includegraphics[width=0.3\textwidth]{figures/ResFit_Spring10QCDFlat_CB_Eta0_ExtrapolatedPar2_PtBin8} \\

    \includegraphics[width=0.3\textwidth]{figures/ResFit_Spring10QCDFlat_CB_Eta0_ExtrapolatedPar2_PtBin9} &
    \includegraphics[width=0.3\textwidth]{figures/ResFit_Spring10QCDFlat_CB_Eta0_ExtrapolatedPar2_PtBin10} & \\
  \end{tabular}
\caption{}
\label{fig:ResFit:App:CB:ExtrapolatedPar2}
\end{figure}


% ----- Crystal Ball Eta0 MCClosure -----
\begin{figure}[ht]
  \centering
  \begin{tabular}{ccc}
    \includegraphics[width=0.3\textwidth]{figures/ResFit_Spring10QCDFlat_CB_Eta0_MCClosure_PtBin0} &
    \includegraphics[width=0.3\textwidth]{figures/ResFit_Spring10QCDFlat_CB_Eta0_MCClosure_PtBin1} &
    \includegraphics[width=0.3\textwidth]{figures/ResFit_Spring10QCDFlat_CB_Eta0_MCClosure_PtBin2} \\

    \includegraphics[width=0.3\textwidth]{figures/ResFit_Spring10QCDFlat_CB_Eta0_MCClosure_PtBin3} &
    \includegraphics[width=0.3\textwidth]{figures/ResFit_Spring10QCDFlat_CB_Eta0_MCClosure_PtBin4} &
    \includegraphics[width=0.3\textwidth]{figures/ResFit_Spring10QCDFlat_CB_Eta0_MCClosure_PtBin5} \\

    \includegraphics[width=0.3\textwidth]{figures/ResFit_Spring10QCDFlat_CB_Eta0_MCClosure_PtBin6} &
    \includegraphics[width=0.3\textwidth]{figures/ResFit_Spring10QCDFlat_CB_Eta0_MCClosure_PtBin7} &
    \includegraphics[width=0.3\textwidth]{figures/ResFit_Spring10QCDFlat_CB_Eta0_MCClosure_PtBin8} \\

    \includegraphics[width=0.3\textwidth]{figures/ResFit_Spring10QCDFlat_CB_Eta0_MCClosure_PtBin9} &
    \includegraphics[width=0.3\textwidth]{figures/ResFit_Spring10QCDFlat_CB_Eta0_MCClosure_PtBin10} & \\
  \end{tabular}
\caption{Closure \mbox{$|\eta|<1.2$}.}
\label{fig:ResFit:App:CB:MCClosure}
\end{figure}

\clearpage


% ----- This should all go to external files ------------
% \section{Determination of a Gaussian resolution}
% \subsection{Conceptual studies using a simple simulation}
% The method presented in the previous section has been studied using a
% simple simulation.

% \begin{figure}[ht]
%   \begin{center}
%      \includegraphics[width=0.45\textwidth]{figures/resFit_ToyMC_PtGenCuts_SpectrumLog}
%    \end{center}
%    \caption{Simple simulation of a sample of ideal dijet events.
%      Generated \pttrue spectrum (histogram) and the underlying pdf (solid line).}
%    \label{fig:resFit:toyMC:ptGenCuts:spectrum}
% \end{figure}

% \begin{figure}[ht]
%   \begin{center}
%     \subfigure[]{
%       \includegraphics[width=0.45\textwidth]{figures/resFit_ToyMC_PtGenCuts_ResolutionBin1}
%    } \subfigure[]{
%       \includegraphics[width=0.45\textwidth]{figures/resFit_ToyMC_PtGenCuts_ResolutionBin7}
%    }
%   \end{center}
%   \caption{Simple simulation of a sample of ideal dijet events.
%     Generated true resolution \mbox{$\ptmeas / \pttrue$} (histogram) and the resolution (solid line) from the maximum likelihood fit in two different \pttrue bins.
%     Note, here ``fit'' does not refer to a fit to the shown histogram but the maximisation of the likelihood~\eqref{eq:resFit:likelihood}.
%   }
%   \label{fig:resFit:toyMC:ptGenCuts:reso}
% \end{figure}

% A sample of 30000 ideal dijet events has been generated with an
% exponentially falling spectrum
% \begin{equation}
%   \label{eq:resFit:toyMCSpec}
%   f\left(\pttrue\right) \propto \exp\left(-\pttrue / \tau\right),
%   \qquad \tau = 80.
% \end{equation}
% ranging from \mbox{$50 < \pttrue < 1000\gev$} (comp. Fig.~\ref{fig:resFit:toyMC:ptGenCuts:spectrum}).
% Two independent measurements of the jet \pt have been simulated
% from a Gaussian resolution
% \begin{equation}
%   \label{eq:resFit:toyMCRes}
%   f_{\vec{b}}\left(\ptmeas|\pttrue\right) = 
%   \frac{1}{\sqrt{2\pi}\sigma}\exp\left[-\frac{1}{2}\left(\frac{\ptmeas - \pttrue}{\sigma}\right)^{2}\right]
% \end{equation}
% (Here and in the following the jet index $i$ has been omitted.)
% The width $\sigma$ has been parameterised as a function of \pttrue and
% the parameters $b_{i}$, \mbox{$i\in\left\{0,2\right\}$}, as
% \begin{equation}
%   \label{eq:resFit:toyMCSigma}
%   \sigma = b_{0}\gev
%   \oplus b_{1}\,\sqrt{\pt\gev}\oplus b_{2}\pt.
% \end{equation}
% The parameter values of the parameters $\vec{b}$ are listed in
% Tab.~\ref{tab:resFit:toyMC:ptGenCuts:fitResult}.
% An example of the simulated resolution is shown in Fig. ~\ref{fig:resFit:toyMC:ptGenCuts:reso}.

% \begin{table}[ht]
%   \centering
%   \begin{tabular}[ht]{lccc}
%     \hline \hline
%     $b_{i}$ & $0$ & $1$ & $2$ \\
%     \hline
%     True value & $4$           & $1.2$           & $0.05$ \\
%     Fit result & $4.5 \pm 0.7$ & $1.18 \pm 0.05$ & $0.051 \pm 0.004$ \\
%     \hline \hline
%   \end{tabular}
%   \caption{Parameter values of the width $\sigma(\pt)$ of the Gaussian
%     resolution. Listed are the true values used for the generation and
%     the fitted values. The uncertainties assigned to the fitted values
%     are the statistical uncertainties from the fit. The parameter
%     correlations are shown in Fig.~\ref{fig:resFit:toyMC:ptGenCuts:parCorr}.}
%   \label{tab:resFit:toyMC:ptGenCuts:fitResult}
% \end{table}

% The jet energy resolution of this dijet sample is to be fitted using the method described above.
% In order to evaluate the dijet pdf~\eqref{eq:resFit:dijetPdf}, the spectrum $f(\pttrue)$ has to be known and the resolution $f_{\vec{b}}(\ptmeasi{i}|\pttrue)$ has to be parameterised appropriately.
% The spectrum is taken directly from the simulation~\eqref{eq:resFit:toyMCSpec}.
% If the same strategy is applied in data, influences of the uncertainty on the simulated spectrum on the fitted resolution have to be considered.
% These are small as shown in Section~\ref{sec:resFit:toyMC:uncert}.
% The resolution is parameterised with a Gaussian, where the width $\sigma$ depends on \pttrue and the parameters $\vec{b}$ as in~\eqref{eq:resFit:toyMCSigma}.
% The dijet pdf is normalised to the \pttrue range of the simulation i.e. $t_{0} = 50\gev$ and $t_{1} = 1000\gev$.
% This setup corresponds to cuts on \pttrue; a more data driven approach is discussed in Section~\ref{sec:resFit:dataDrivenExt}.

% \begin{figure}[ht]
%   \centering
%   \includegraphics[width=0.45\textwidth]{figures/resFit_ToyMC_PtGenCuts_Correlations}
%   \caption{Correlation coefficients of the fitted parameter values
%     $\vec{b}$ of the width $\sigma$ of the Gaussian
%     resolution. Parameter $3$ corresponds to the slope $\tau$ of the
%     spectrum, which is fixed during the fit, and is to be ignored.}
%   \label{fig:resFit:toyMC:ptGenCuts:parCorr}
% \end{figure}

% The fitted parameter values $\vec{b}$ are listed in Tab.~\ref{tab:resFit:toyMC:ptGenCuts:fitResult}.
% They agree with the true values within the statistical uncertainties.

% The parameters are strongly (anti-) correlated (comp. Fig.~\ref{fig:resFit:toyMC:ptGenCuts:parCorr}).
% In the present \pttrue interval from \mbox{$t_{0} = 50\gev$} to \mbox{$t_{1} = 1000\gev$}, the used parameterisation of the Gaussian width $\sigma$ is over-determined.
% The $b_{0}$ term in~\eqref{eq:resFit:toyMCSigma} is most important at very low \pt while the $b_{2}$ term dominats at very large \pt.
% Hence, omitting either the terms with $b_{0}$ and $b_{2}$ or the term with $b_{1}$ in~\eqref{eq:resFit:toyMCSigma} would have been sufficient to describe the measured events.

% \begin{figure}[ht]
%   \begin{center}
%     \subfigure[]{
%       \includegraphics[width=0.45\textwidth]{figures/resFit_ToyMC_PtGenCuts_Sigma}
%   } \subfigure[]{
%       \includegraphics[width=0.45\textwidth]{figures/resFit_ToyMC_PtGenCuts_SigmaRelDifference}
%   }
%   \end{center}
%   \caption{(a) Relative Gaussian width $\sigma(\pt)/\pt$ evaluated with the fitted
%     parameter values (solid line) in comparison to the generated width
%     (dashed line).  (b) shows
%     the relative difference of the two curves. The shaded area represents the propagated statistical
%     uncertainty on the fitted parameter values, taking into account the
%     parameter correlations.}
%   \label{fig:resFit:toyMC:ptGenCuts:sigma}
% \end{figure}

% The relative Gaussian width $\sigma(\pt)/\pt$ evaluated with the fitted
% parameter values is shown in Fig.~\ref{fig:resFit:toyMC:ptGenCuts:sigma}
% in comparison to the true width at generation.
% There is good agreement between the fitted and the true resolution
% within the statistical uncertainties.
% The uncertainties are about $1\%$ at low \pt rising to
% about $3\%$ at $\pt \approx 600\gev$ where there is sufficient statistics in
% the generated sample (comp. Fig~\ref{fig:resFit:toyMC:ptGenCuts:spectrum}).
% For larger \pt the uncertainties rise up to $4.5\%$.

% An example of the resulting resolution in comparison to the true
% resolution is shown in Fig.~\ref{fig:resFit:toyMC:ptGenCuts:reso} for
% one \pttrue bin.


% \subsection{Extension to a data driven event selection}\label{sec:resFit:dataDrivenExt}

% So far, events have been selected by cuts on the true jet \pt.
% In a more data driven approach the selection is to be applied to the measured jet \pt.
% Events are selected, if for one of the two jets
% \begin{equation*}
%   \ptmin < \ptmeas < \ptmax.
% \end{equation*}
% Each event is considered twice; once for a \pt cut on the first and once for a \pt cut on the second jet.
% In this way any bias in the selection arising e.g. from always selecting the first and thus the jet fluctuated upwards is avoided and the available statistics is doubled.
% Since the \pt of the other jet can fluctuate arbitrarily, sensitivity to the resolution and in particular the non-Gaussian tails is assured; this is not the case when cutting e.g. on \mbox{$\ptdijet = \frac{1}{2}(\pt^{1}+pt^{2})$}, the average dijet \pt, as demonstrated in Fig.~\ref{fig:resFit:ptBinning}.

% \begin{figure}[ht]
%   \centering
%   \includegraphics[width=0.45\textwidth]{figures/resFit_QCD_PtBinningComp}
%   \caption{Dependence of the true resolution on the definition of the quantity the cuts are applied to.
%     Cuts are applied either to \mbox{$\ptdijet = \frac{1}{2}(\pt^{1}+pt^{2})$} (open markers) or to the \pt of one of the jets while the other fluctuates arbitrarily (solid markers).
%     The non-Gaussian tails are surpressed when cutting on \ptdijet.}
%   \label{fig:resFit:ptBinning}
% \end{figure}

% Cutting on the measured \pt leads to migration effects at the edges of the selected \pt range due to the finite jet resolution.
% Considering the lower limit \ptmin, in an event with \mbox{$\pttrue < \ptmin$} the actually measured jet \pt might fluctuate upwards resulting in \mbox{$\ptmeas > \ptmin$} and thus false acceptance of the event.
% Similarly, events might be falsely rejected due to downward fluctuations.
% Since the dijet cross section is steeply falling with \pt, the described migration is asymmetric i.e. there is more upward than downward migration and more events are falsely accepted.
% (Analogue considerations hold true for the upper limit.)
% The described migration effects are demonstrated in Fig.~\ref{fig:resFit:toyMC:ptCuts:spectrum}.

% \begin{figure}[ht]
%   \begin{center}
%     \subfigure[]{
%       \includegraphics[width=0.45\textwidth]{figures/resFit_ToyMC_PtCuts_SpectrumLog}
%     } \subfigure[]{
%       \includegraphics[width=0.45\textwidth]{figures/resFit_ToyMC_PtCuts_SpectrumLinear}
%     }
%   \end{center}
%   \caption{Simple simulation of a sample of ideal dijet
%     events. Generated \pttrue spectrum (histogram) and the underlying pdf (solid
%     line) after cuts on measured \pt. The spectrum is shown (a) in log
%     scale and (b) in linear scale.}
%   \label{fig:resFit:toyMC:ptCuts:spectrum}
% \end{figure}

% \begin{figure}[ht]
%   \begin{center}
%     \subfigure[]{
%       \includegraphics[width=0.45\textwidth]{figures/resFit_ToyMC_PtCuts_ResolutionBin1}
%    } \subfigure[]{
%       \includegraphics[width=0.45\textwidth]{figures/resFit_ToyMC_PtCuts_ResolutionBin7}
%    }
%   \end{center}
%   \caption{Simple simulation of a sample of ideal dijet
%     events. Generated true resolution \mbox{$\ptmeas / \pttrue$} (histogram) and the fitted
%     resolution (solid line) in two different \pttrue bins after cuts on measured \pt.
%     Note the migration effects in (a): the narrow peak at the right side is due to the jet that has been cut on and hence these are jets fluctuating upwards into the \pt bin.
%     The wider peak centered around one is due to the other jet that fluctuates unconstrained.
%     (There is more upward than downward fluctuation due to the falling spectrum.)
%   }
%   \label{fig:resFit:toyMC:ptCuts:reso}
% \end{figure}

% \begin{figure}[ht]
%   \begin{center}
%     \subfigure[]{
%       \includegraphics[width=0.45\textwidth]{figures/resFit_ToyMC_PtCuts_Sigma}
%   } \subfigure[]{
%       \includegraphics[width=0.45\textwidth]{figures/resFit_ToyMC_PtCuts_SigmaRelDifference}
%   }
%   \end{center}
%   \caption{(a) Relative Gaussian width $\sigma(\pt)/\pt$ evaluated with the fitted
%     parameter values (solid line) in comparison to the generated width
%     (dashed line). The fit has been performed after cuts on measured \pt. (b) shows
%     the relative difference of the two curves. The shaded area represents the propagated statistical
%     uncertainty on the fitted parameter values, taking into account the
%     parameter correlations.}
%   \label{fig:resFit:toyMC:ptCuts:sigma}
% \end{figure}

% The dijet pdf~\eqref{eq:resFit:dijetPdf} is adapted to include the cuts on the measured jet \pt.
% First, the allowed range of measured \pt is restricted for the jet the \pt cut is applied to.
% In the following, it is assumed that this is the first jet; the pdfs of measured jet \pt therefore become
% \begin{eqnarray*}
%   f_{\vec{b}}\left(\ptmeasi{1}|\pttrue\right) & = & 
%   \frac{1}{\mathcal{N}_{1}}\exp\left[-\frac{1}{2}\left(\frac{\ptmeasi{1}
%         - \pttrue}{\sigma}\right)^{2}\right]
%   \cdot\theta\left(\ptmeasi{1} - \ptmin\right)
%   \cdot\theta\left(\ptmax - \ptmeasi{1}\right) \\
%   f_{\vec{b}}\left(\ptmeasi{2}|\pttrue\right) & = & 
%   \frac{1}{\mathcal{N}_{2}}\exp\left[-\frac{1}{2}\left(\frac{\ptmeasi{2}
%         - \pttrue}{\sigma}\right)^{2}\right] \\
% \end{eqnarray*}
% with the normalisation constants
% \begin{eqnarray*}
%   \mathcal{N}_{1} & = &
%   \int^{\ptmax}_{\ptmin}\dif{\ptmeasi{1}}\,f_{\vec{b}}\left(\ptmeasi{1}|\pttrue\right)
%   = \sqrt{\frac{\pi}{2}}\sigma \left[ \text{erf}\left(\frac{\ptmax -
%         \pttrue}{\sqrt{2}\sigma}\right) - \text{erf}\left(\frac{\ptmin
%         - \pttrue}{\sqrt{2}\sigma}\right)\right] \\
%   \mathcal{N}_{2} & = &
%   \int^{\infty}_{0}\dif{\ptmeasi{2}}\,f_{\vec{b}}\left(\ptmeasi{2}|\pttrue\right)
%   = \sqrt{\frac{\pi}{2}}\sigma \left[ 1 +
%     \text{erf}\left(\frac{\pttrue}{\sqrt{2}\sigma}\right)\right]
%   \approx \sqrt{2\pi}\sigma.
% \end{eqnarray*}
% Second, the pdf of the true \pt is extended to incorporate the events
% migrating into the allowed \pt range by
% \begin{equation}
%   \label{eq:resFit:toyMC:ptCuts:extendedSpectrum}
%   \tilde{f}\left(\pttrue\right) = \frac{1}{\mathcal{N}_{\tilde{f}}}
%   f\left(\pttrue\right) \int^{\ptmax}_{\ptmin}\dif{x}\,f_{\text{MC}}\left(x|\pttrue\right)
% \end{equation}
% with the normalisation
% \begin{equation*}
%   \mathcal{N}_{\tilde{f}} = \int^{\infty}_{0}\dif{\pttrue}\,
%   f\left(\pttrue\right) \int^{\ptmax}_{\ptmin}\dif{x}\,f_{\text{MC}}\left(x|\pttrue\right).
% \end{equation*}
% Here, $f(\pttrue)$ denotes the underlying spectrum~\eqref{eq:resFit:toyMCSpec} and
% $f_{\text{MC}}(x|\pttrue)$ the pdf of measured \pt.
% $f_{\text{MC}}(x|\pttrue)$ in \eqref{eq:resFit:toyMC:ptCuts:extendedSpectrum} is taken from the
% simulation and the parameters are kept fixed during the maximisation.
% Thus usage of the simulation introduces a further uncertainty but they
% are small as in case of the underlying spectrum as shown below.
% The adapted dijet pdf then becomes
% \begin{equation}
%   \label{eq:resFit:toyMC:ptCuts:pdf}
%   f_{\vec{b}}\left(\ptmeasi{1},\ptmeasi{2}\right) = \int^{\infty}_{0}\dif{\pttrue}\,\tilde{f}\left(\pttrue\right)
%   \cdot f_{\vec{b}}\left(\ptmeasi{1}|\pttrue\right)
%   \cdot f_{\vec{b}}\left(\ptmeasi{2}|\pttrue\right),
% \end{equation}
% which is properly normalised to
% \begin{equation*}
%   1 = \int^{\ptmax}_{\ptmin}\dif{\ptmeasi{1}}\,\int^{\infty}_{0}\dif{\ptmeasi{2}}\, f_{\vec{b}}\left(\ptmeasi{1},\ptmeasi{2}\right).
% \end{equation*}

% \begin{table}[ht]
%   \centering
%   \begin{tabular}[ht]{lccc}
%     \hline \hline
%     $b_{i}$ & $0$ & $1$ & $2$ \\
%     \hline
%     True value & $4$           & $1.2$                   & $0.05$ \\
%     Fit result   & $4 \pm 1$ & $1.20 \pm 0.07$ & $0.049 \pm 0.005$ \\
%     \hline \hline
%   \end{tabular}
%   \caption{Parameter values of the width $\sigma(\pt)$ of the Gaussian
%     resolution. Listed are the true values used for the generation and
%     the fitted values. The fit has been performed after cuts on measured \pt.
%     The uncertainties assigned to the fitted values
%     are the statistical uncertainties from the fit; correlations are not shown.}
%   \label{tab:resFit:toyMC:ptCuts:fitResult}
% \end{table}

% Again the sample of ideal dijet events is used to test the discussed extension of the method.
% Cuts are placed on the measured \pt at \mbox{$\ptmin = 80\gev$} and \mbox{$\ptmax = 800\gev$} (comp. Fig.~\ref{fig:resFit:toyMC:ptCuts:spectrum}).
% The jet energy resolution of the selected dijet events is fitted using the modified pdf~\eqref{eq:resFit:toyMC:ptCuts:pdf}.
% Uncertainties due to uncertainties in the description of the spectrum are discussed in Section~\ref{sec:resFit:toyMC:uncert}.

% The fitted parameter values $\vec{b}$ of the Gaussian width $\sigma$ are listed in Tab.~\ref{tab:resFit:toyMC:ptCuts:fitResult}.
% They agree with the true values within the statistical uncertainties.
% As expected, they feature the same correlation pattern as in case of \pttrue cuts.

% The relative Gaussian width $\sigma(\pt)/\pt$ evaluated with the fitted parameter values is shown in Fig.~\ref{fig:resFit:toyMC:ptCuts:sigma} in comparison to the true width at generation.
% There is good agreement between the fitted and the true resolution within the statistical uncertainties.
% As before, the uncertainties are about $1\%$ at low \pt rising to about $3\%$ at $\pt \approx 600\gev$ and up to $4.5\%$ at very large \pt.
% An example of the resulting resolution in comparison to the true resolution is shown in Fig.~\ref{fig:resFit:toyMC:ptCuts:reso} for one \pttrue bin.

% It can be concluded therefore that the method is working also for a data driven event selection.



% \subsection{Estimation of systematic uncertainties}\label{sec:resFit:toyMC:uncert}

% The dijet pdf discussed above contains knowledge of the actual dijet spectrum and --- in case of cuts on measured jet \pt --- also of the resolution to describe the cut-off effects.
% When applying the method to data this information would be taken from a MC simulation.
% Test have been performed to evaluate the influence of uncertainties in the MC description on the fitted resolution.

% \begin{figure}[ht]
%   \centering
%   \includegraphics[width=0.45\textwidth]{figures/resFit_ToyMC_PtCuts_SigmaUncertainties}
%   \caption{Relative differences of the Gaussian widths $\sigma(\pt)$ evaluated with the fitted parameter values for different variations of the spectrum and the correct spectrum.
%     The variation has been applied to the slope parameter $\tau$ in \eqref{eq:resFit:toyMCSpec} (``Spectrum'') as well as all three parameters $b_{\text{MC},i}$ in \eqref{eq:resFit:toyMC:ptCuts:extendedSpectrum} (``Resolution'').}
%   \label{fig:resFit:toyMC:uncert:systUncertainties}
% \end{figure}

% \begin{table}[ht]
%   \centering
%   \begin{tabular}[ht]{lccc}
%     \hline \hline
%     $b_{i}$ & $0$ & $1$ & $2$ \\
%     \hline
%     True value         & $4$       & $1.2$           & $0.05$ \\
%     Correct            & $4 \pm 1$ & $1.20 \pm 0.07$ & $0.049 \pm 0.005$ \\
%     Spectrum $+30\%$   & $3 \pm 2$ & $1.22 \pm 0.07$ & $0.049 \pm 0.005$ \\
%     Spectrum $-30\%$   & $5 \pm 1$ & $1.19 \pm 0.08$ & $0.052 \pm 0.006$ \\
%     Resolution $+30\%$ & $5 \pm 1$ & $1.13 \pm 0.07$ & $0.054 \pm 0.004$ \\
%     Resolution $-30\%$ & $5 \pm 2$ & $1.19 \pm 0.08$ & $0.050 \pm 0.006$ \\
%     \hline \hline
%   \end{tabular}
%   \caption{Parameter values of the width $\sigma(\pt)$ of the Gaussian resolution.
%     Listed are the true values used for the generation and the fitted values for different variations of the spectrum.
%     The uncertainties assigned to the fitted values are the statistical uncertainties from the fit.}
%   \label{tab:resFit:toyMC:uncert:fitResult}
% \end{table}

% The fit has been repeated with the same setup as described in Section~\ref{sec:resFit:dataDrivenExt}.
% In order to evaluate the influence of the spectrum on the result the slope i.e. the parameter $\tau$ in \eqref{eq:resFit:toyMCSpec} has been varied by $\pm30\%$.
% Cut-off effects in the spectrum are described by the true resolution $f_{\text{MC}}(x|\pttrue)$ in \eqref{eq:resFit:toyMC:ptCuts:extendedSpectrum}.
% All three parameters $b_{\text{MC},i}$ of $f_{\text{MC}}(x|\pttrue)$ have been varied by $\pm30\%$ to evaluate uncertainties arising from the simulation.

% The fitted parameter values for each variation are listed in Tab.~\ref{tab:resFit:toyMC:uncert:fitResult} in comparion to the values obtained when using the correct spectrum.
% The relative differences of the Gaussian widths $\sigma$ calculated with these fitted parameter values to the $\sigma$ using the correct spectrum are presented in Fig.~\ref{fig:resFit:toyMC:uncert:systUncertainties} as a function of \pt for the different variations; they are below $4\%$.
% A varied spectrum predominantly results in a larger resolution.



% \subsection{Application to a realistic QCD simulation}\label{sec:resFit:qcdSelection}
% The method has then been applied to QCD events with 7\tev center of
% mass energy which were generated with PYTHIA and went through a full
% GEANT4 based CMS detector simulation\footnote{The used dataset is
%   \texttt{/QCDFlat\_Pt15to3000/Summer09-MC\_31X\_V9\_7TeV-v1/GEN-SIM-RECO}}.
% Jets have been reconstructed from calorimeter towers using the
% anti-$k_{T}$ algorithm with size parameter $R=0.5$.
% The jet energy scale has been corrected for $\eta$ and \pt dependence
% by the appropriate L2L3 corrections\footnote{These are the
%   \texttt{JetMETCorrections.Configuration.L2L3Corrections\_Summer09\_7TeV\_ReReco332\_cff}
% corrections.}.

% \begin{figure}[ht]
%   \begin{center}
%     \subfigure[]{
%       \label{fig:resFit:qcd:dijetspectrum:subA}
%       \includegraphics[width=0.45\textwidth]{figures/resFit_QCD_MCSpectrum}
%     } \subfigure[]{
%       \label{fig:resFit:qcd:dijetspectrum:subB}
%       \includegraphics[width=0.45\textwidth]{figures/resFit_QCD_MCTruthResolution}
%     }
%   \end{center}
%   \caption{(a) \ptparticle distribution and (b) MC truth resolution of the leading two jets in the selected dijet sample.
%     The solid lines are fits of the indicated functions.}
%   \label{fig:resFit:qcd:dijetspectrum}
% \end{figure}

% In the following jets are considered to be ordered in corrected \ptreco.
% Dijet events are selected by requiring the third jet to have small \pt compared to the leading two jets,
% \begin{enumerate}
% \item $\ptrel < 0.1$ with $\ptrel = \frac{2\ptsub{3}}{\ptsub{1} + \ptsub{2}}$.
% \end{enumerate}
% This selection cut has been adopted from~\cite{CMSAN-2008/031}.
% The additional cut on $\Delta\phi$ performed there has been dropped as it has been found to be strongly correlated to the cut on \ptrel.
% In order to reject jets clustered from calorimeter noise a cut on the electromagnetic fration of the two leading jets is applied,
% \begin{enumerate}
% \item[2.] $0.01 < f_{\text{em},i} < 0.99$ with $i\in\{1,2\}$.
% \end{enumerate}

% So far and in the following, the resolution is defined as a function of jet \pt.
% It is influenced for example by the magnetic solenoid field which bends charged particles out of the jet cone depending on their \pt.
% Other effects such as the intrinsic calorimeter resolution are energy $E$ dependent.
% The dependence on the difference between the $E$ and \pt is minimised by considering dijet events with both leading jets in a restricted $\eta$ region,
% \begin{enumerate}
% \item[3.] $|\eta_{i}| < 1.2$ with $i\in\{1,2\}$.
% \end{enumerate}
% It is planned to apply the method to different $\eta$ regions taking into account also those events in which the two leading jets are in different $\eta$ regions.

% The events are weighted corresponding to an integrated luminosity of $100\,\text{pb}^{-1}$.

% The resulting \ptparticle spectrum is shown in Fig.~\ref{fig:resFit:qcd:dijetspectrum:subA}.
% It is fitted with an empiric function
% \begin{equation}
%   \label{eq:resFit:qcd:ptGenSpectrum}
%   f\left(\ptparticle\right) = \sum^{2}_{i=0}\,\exp\left[-\left(a_{2i} + a_{2i+1}\ptparticle\right)\right]
% \end{equation}
% to have a first applicable parameterisation of the spectrum $f(\pttrue)$.
% The fitted parameter values $a_{i}$ are listed in Table~\ref{tab:resFit:qcd:dijetspectrum} and the resulting pdf is superimposed in Fig.~\ref{fig:resFit:qcd:dijetspectrum:subA}.
% \begin{table}[ht]
%   \centering
%   \begin{tabular}{rc}
%     \hline
%     \hline
%     $a_{i}$ & Fitted value \\
%     \hline
%     $0$ & $0.56 \pm 0.04$ \\
%     $1$ & $0.0300 \pm 0.0004$ \\
%     $2$ & $3.91 \pm 0.09$ \\
%     $3$ & $0.0152 \pm 0.0004$ \\
%     $4$ & $7.15 \pm 0.05$ \\
%     $5$ & $0.00837 \pm 0.00008$ \\
%     \hline
%     \hline
%   \end{tabular}
%  \caption{Fitted parameter values of the dijet \ptparticle spectrum.}
%   \label{tab:resFit:qcd:dijetspectrum}
% \end{table}

% \begin{figure}[ht]
%   \centering
%   \includegraphics[width=0.45\textwidth]{figures/resFit_PtDependentSigma}
%   \caption{Width $\sigma$ of a Gaussian resolution in QCD dijet events. Shown is $\sigma$ from fitted parameter values (red line) in comparison to the truth from the MC simulation (blue line).}
%   \label{fig:resFit:qcd:ptDependentSigma}
% \end{figure}

% The determination of a Gaussian resolution~\eqref{eq:resFit:toyMCRes} with a \pt dependent width $\sigma$ as in~\eqref{eq:resFit:toyMCSigma} has been performed as for the ideal sample above.
% The results are not satisfying (comp. Fig.~\ref{fig:resFit:qcd:ptDependentSigma}).

% Various test have been performed and the method appears to work fine for a moderately falling \pt spectrum.
% In case of a realistic QCD spectrum (comp. Fig.~\ref{fig:resFit:qcd:dijetspectrum:subA}), however, the
% determined resolution does not describe the true resolution at large \pt anymore.
% The failure of the method is assumed to be due to the non-ideal topology of the selected dijet events, namely the presence of a third jet.
% These effects and an extension of the presented method are under study.
% Meanwhile a modified strategy to determine the resolution is investigated and presented in the following sections.


% \section{Determination of the mean Gaussian resolution in \pt bins}
% \subsection{Strategy}
% The mean Gaussian resolution with constant width $\bar{\sigma}$ is to be determined in bins of \pt.
% Afterwards it is interpolated between the bins with a continuous function.
% The bin size is chosen as a compromise between having small bins for a differential measurement and sufficient statistics in each bin.

% Dijet events are selected by the cuts described in Section~\ref{sec:resFit:qcdSelection} from the same QCD sample.
% Bins are defined by cuts on the measured jet \pt as described in Section~\ref{sec:resFit:dataDrivenExt}.
% Table~\ref{tab:resFit:qcd:ptBins} lists the chosen binning.

% \begin{table}[ht]
%   \centering
%   \begin{tabular}{rcc}
%     \hline
%     \hline
%     Bin & \ptmin (\gev) & \ptmax (\gev) \\
%     \hline
%     0 & 100 & 120 \\
%     1 & 120 & 140 \\
%     2 & 140 & 170 \\
%     3 & 170 & 200 \\
%     4 & 200 & 250 \\
%     5 & 250 & 300 \\
%     6 & 300 & 400 \\
%     7 & 400 & 600 \\
%     8 & 600 & 1000 \\
%     \hline
%     \hline
%   \end{tabular}
%   \caption{Definition of \pt bins for the fit of the mean Gaussian resolution.}
%   \label{tab:resFit:qcd:ptBins}
% \end{table}

% The modified spectrum $\tilde{f}(\pttrue)$ from~\eqref{eq:resFit:toyMC:ptCuts:extendedSpectrum} is utilised in the fit.
% The underlying spectrum $f(\pttrue)$ is described by~\eqref{eq:resFit:qcd:ptGenSpectrum} with the parameter values listed in Tab.~\ref{tab:resFit:qcd:dijetspectrum}.
% Cut-off effects are included as before using the true -- and \pt dependent -- resolution $f_{\text{MC}}(x|\pttrue)$ in $\tilde{f}(\pttrue)$.
% $f_{\text{MC}}(x|\pttrue)$ has been determined by fitting the $\ptreco/\ptparticle$ distributions in small $\ptparticle$ bins with a central Gaussian.
% The widths of the Gaussians are shown in Fig.~\ref{fig:resFit:qcd:dijetspectrum:subB}.
% They are fitted with the continous function
% \begin{equation}
%   \label{eq:resFit:qcd:sigma}
%   \frac{\sigma}{\pt} = \frac{b_{1}\sqrt{\gev}}{\sqrt{\pt}} \oplus \frac{b_{2}\gev}{\pt}.
% \end{equation}
% (In this \pt range there is no sensitivity to a third term $b_{0}$, which is why it is omitted.)
% The values $b_{i}$ of the true resolution\footnote{The fit is applied once to the presented Fig.~\ref{fig:resFit:qcd:dijetspectrum:subB} and once to an analogue plot with a logarithmic binning in \ptparticle.
% The parameters listed in Tab.~\ref{tab:resFit:qcd:resolution} are the mean values from the two fits.}
% are listed in Tab.~\ref{tab:resFit:qcd:resolution}.

% Figure~\ref{fig:resFit:qcd:specExBin} shows the \ptparticle spectrum in an example bin.
% It is well described by $\tilde{f}(\pttrue)$.


% \subsection{Results for a QCD simulation}
% \begin{figure}[ht]
%   \begin{center}
%     \subfigure[]{
%       \label{fig:resFit:qcd:specExBin}
%       \includegraphics[width=0.45\textwidth]{figures/resFit_QCD_Gauss_Spectrum_PtBin3}
%     } \subfigure[]{
%       \label{fig:resFit:qcd:extrapolationExBin}
%       \includegraphics[width=0.45\textwidth]{figures/resFit_QCD_Gauss_ExtrapolatedSigma_PtBin3}
%     }
%   \end{center}
%   \caption{(a) \ptparticle distribution and of the leading two jets in an example \pt bin.
%     The solid line is the spectrum $\tilde{f}(\pttrue)$.
%     (b) Fitted Gaussian mean widths $\bar{\sigma}/\pt$ for different cuts on \ptrel in the same example \pt bin.
%     The solid line is a linear fit to extrapolate $\bar{\sigma}/\pt$ for ideal dijet events.}
% \end{figure}

% The fit has been performed for the listed \pt bins.
% The result is affected by the presence of a third jet.
% For example, the mean resolution for jets with \mbox{$170 < \ptreco < 200\gev$} has been determined to \mbox{$\bar{\sigma}/\bar{\pt} = 0.136 \pm 0.002$} whereas the MC truth value is \mbox{$0.0933$}.
% This difference is dependent on the maximum \ptrel as demonstrated in Fig.~\ref{fig:resFit:qcd:extrapolationExBin}, which shows the fitted mean resolution for different values of \ptrel.
% There is a linear trend visible.
% In order to extrapolate to the results to the case of an ideal dijet event, the $\bar{\sigma}\/\pt$ are fitted with a linear function and the y axis intercept is considered as the correct jet energy resolution.

% The extrapolated Gaussian mean widths $\bar{\sigma}/\pt$ found in this way are shown for different \pt bins in Fig.~\ref{fig:resFit:qcd:extrapolation}.
% Here, the \pt values are the mean values of the assumed spectra $\tilde{f}(\pttrue)$ in each bin.
% The shown statistical uncertainties are combined from two sources.
% \begin{enumerate}
% \item The first contribution is the statistical uncertainty on the parameter of y axis intercept in the linear extrapolation (comp. Fig.~\ref{fig:resFit:qcd:extrapolationExBin}).
% \item The second contribution arises from the uncertainty due to the statistics in the used MC sample.
%   As the event weights are large for small \pt, statistical fluctations might become significant.
%   In order to roughly evaluate this uncertainty, the same fits have been performed again without event weights (for the used sample this corresponds to a flattened QCD spectrum).
%   The statistical uncertainties obtained in this case on the y axis intercept are added in quadrature to the ones in 1.
%   This procedure has to be improved to obtain a statistically correct description of the uncertainty.
% \end{enumerate}

% The dashed line in Fig.~\ref{fig:resFit:qcd:extrapolation} represents the MC truth resolution.
% The fitted values are in reasonable agreement with the MC truth.
% They are then fitted with the continous function \eqref{eq:resFit:qcd:sigma} (solid line).
% The parmeters of this fit are listed in Tab.~\ref{tab:resFit:qcd:resolution}.

% \begin{table}[ht]
%   \centering
%   \begin{tabular}[ht]{lcc}
%     \hline \hline
%     $b_{i}$ & $1$ & $2$ \\
%     \hline
%     MC truth    & $1.145 \pm 0.001$ & $0.0370 \pm 0.0006$ \\
%     Fit result  & $1.18  \pm 0.04$  & $0.032  \pm 0.007$ \\
%     \hline \hline
%   \end{tabular}
%   \caption{Parameter values of the width $\sigma(\pt)$ of the Gaussian resolution~\eqref{eq:resFit:qcd:sigma}.
%     Listed are the MC truth values and the fitted values.
%     For the fit, the mean $\bar{\sigma}/\pt$ has been fitted in different \pt bins and the result fitted again with the continous function~\eqref{eq:resFit:qcd:sigma} (comp. Fig.~\ref{fig:resFit:qcd:extrapolation}).
%     The uncertainties assigned to the fitted values are the statistical uncertainties.}
%   \label{tab:resFit:qcd:resolution}
% \end{table}

% \begin{figure}[ht]
%   \begin{center}
%     \subfigure[]{
%       \label{fig:resFit:qcd:extrapolation}
%       \includegraphics[width=0.45\textwidth]{figures/resFit_QCD_Gauss_ExtrapolatedResolution}
%     } \subfigure[]{
%       \label{fig:resFit:qcd:systUncert}
%       \includegraphics[width=0.45\textwidth]{figures/resFit_QCD_Gauss_SystematicUncertainties}
%     }
%   \end{center}
%   \caption{(a) Gaussian mean widths $\bar{\sigma}/\pt$ from extrapolation \mbox{$\ptrel\rightarrow0$} in different \pt bins.
%   The \pt values are the mean values of the assumed spectra $\tilde{f}(\pttrue)$ in each bin.
%   The error bars indicate the combined statistical uncertainties from the extrapolation and the MC statistics.
%   The solid line is a fit to the $\bar{\sigma}/\pt$, the dashed line shows the MC truth resolution for comparison.
%   (b) Estimation of systematic uncertainties.
%   The lines represent relative differences of the resolution for different variations of the spectrum to the correct spectrum.}
% \end{figure}
% \clearpage


% \section{Determination of a non-Gaussian resolution}

% \subsection{\mht spectrum from Gaussian resolution}
% In order to evaluate the influence of non-Gaussian tails in the resolution to the \mht spectrum, a simple generator level based jet smearing has been performed.
% The same dijet selection as presented in Sec.~\ref{sec:resFit:qcdSelection} has been applied.
% For this demonstration, the \ptparticle of the leading two jets has been weighted with a random number from a Gaussian resolution pdf, where $\sigma$ has been parameterised using the MC truth values listed in Tab.~\ref{tab:resFit:qcd:resolution}.

% \begin{figure}[ht]
%   \begin{center}
%     \subfigure[]{
%       \includegraphics[width=0.45\textwidth]{figures/MHTSpectrumGauss}
%     } \subfigure[]{
%       \includegraphics[width=0.45\textwidth]{figures/MHTRatioGauss}
%     }
%   \end{center}
%   \caption{(a) \mht from the two leading jets in a QCD dijet sample (marker) and prediction from smearing of the corresponding generator level jets with the true Gaussian resolution (line).
%     (b) Ratio of prediction and truth.}
% \label{fig:resFit:qcd:mhtGauss}
% \end{figure}

% In Fig.~\ref{fig:resFit:qcd:mhtGauss} the \mht spectrum resulting from
% the weighted \ptparticle of the two jets is compared to the \mht spectrum from the corresponding \ptreco.
% The actual \mht is underestimated by about $5\%$ up to 200\gev.
% Above, which are $\approx2\%$ of all the events, the \mht is underestimated by $\approx50\%$.
% This difference is explained by the presence of non-Gaussian tails in the resolution which were not considered.

% In conclusion, it is necessary to include the description of non-Gaussian tails of the resolution into the fit in order to accurately predict the \mht spectrum using a smearing method.


% \subsection{Parameterisation of the resolution by a Crystal Ball function}
% \textit{to be added}



% ----- Bibliography ------------------------------------

\begin{thebibliography}{9}
\bibitem{bib:akj} M.~Cacciari, G.~P.~Salam, and G.~Soyez,
  \textit{The anti-kt jet clustering algorithm},
  JHEP~0804:063 (2008)
\bibitem{bib:cmspas:jec} The CMS Collaboration,
  \textit{Plans for Jet Energy Corrections at CMS},
  CMS PAS JME-07-002 (2007)
\bibitem{bib:cmspas:jetid}  The CMS Collaboration,
  \textit{Calorimeter Jet Quality Criteria for the First CMS Collision Data},
  CMS PAS JME-09-008 (2008)
\bibitem{bib:cmsan:mcjer} R.~Ciesielski et al.,
  \textit{Jet Energy Resolutions Derived from QCD Simulation for the Analysis of First $\sqrt{s}=7\,\mathrm{TeV}$ Collision Data},
  CMS AN 2010/121 (2010)
\bibitem{bib:cmspas:dijetasymm} The CMS Collaboration,
  \textit{Measurement of the Jet Energy Resolutions and Jet Reconstruction Efficiency at CMS},
  CMS PAS JME-09-007 (2009)
\end{thebibliography}



\end{document}
