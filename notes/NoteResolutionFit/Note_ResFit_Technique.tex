% $Id: Note_ResFit_Technique.tex,v 1.14 2010/10/08 07:52:28 mschrode Exp $


\section{Description of the Method}\label{sec:ResFit:Method}

\subsection{Definition of the Dijet Probability Density}
In case of ideal dijet events with exactly two jets in the final
state, the transverse momenta of the original partons are balanced due
to momentum conservation.
If hadronisation and jet clustering effects are neglected, also the
\ptgen of the subsequent particle level jets will be equal.
Assuming further that detector response and resolution effects on both jets are not correlated with each other, the probability density (\textit{pdf}) of a dijet event is defined as
\begin{equation}
  \label{eq:ResFit:DijetPdf}
  g_{\mathbf{\xi}}\left(\pti{1},\pti{2}\right) \propto \int\dif{\pttrue}\,f\left(\pttrue\right)
  \cdot r_{\mathbf{\xi}}\left(\pti{1}|\pttrue\right)
  \cdot r_{\mathbf{\xi}}\left(\pti{2}|\pttrue\right) \; ,
\end{equation}
where \pti{i} and \pttrue are the absolute transverse momenta of the
$i$-th jet, \mbox{$i = 1,2$}, at detector level and at particle level, respectively.
Note that the notation \pttrue instead of \ptgen has been chosen in order to emphasise the assumed \pt
balance at particle level, whereas in general \mbox{$\ptgeni{1} \neq \ptgeni{2}$} due to the above
mentioned effects.
Furthermore, $f$ denotes the pdf of \pttrue, \ie the particle level differential jet 
cross section.
In the following, $f$ is taken from the MC
simulation\footnote{The arising systematic uncertainty is evaluated in
  \qsec{sec:ResFit:Systematics}.}
and kept fixed during the maximisation procedure.
$r_{\mathbf{\xi}}(\pti{i}|\pttrue)$ denotes the pdf to measure \pti{i} for the
$i$-th jet given \pttrue.
It is a function that depends on the free parameter set $\mathbf{\xi}$.

It is important to note that $g$ does not depend on \pttrue.
The latter is an integration variable and hence all possible values of \pttrue
contribute to $g$ with the individual weights $f(\pttrue)$.

For a sample of $N$ dijet events, a likelihood is defined as
\begin{equation}
  \label{eq:ResFit:Likelihood}
  \mathcal{L}\left(\mathbf{\xi}\right) = \prod^{N}_{k=1} g_{\mathbf{\xi},k}\left(\pti{1},\pti{2}\right).
\end{equation}
The values of $\mathbf{\xi}$ which maximise $\mathcal{L}$ correspond
to the response $r_{\mathbf{\xi}}$ that describes the data best.
The pdf of the jet \pt response \mbox{$R = \pt / \pttrue$} is obtained by parameter transformation
\begin{equation*}
  r_{\mathbf{\xi}}\left(R|\pttrue\right) =
  r_{\mathbf{\xi}}\left(\pt\left(R\right)|\pttrue\right)\cdot\left|\frac{\dif{\pt}}{\dif{R}}\right|.
\end{equation*}


\subsection{Case of a Gaussian Response}

No assumption about the functional form of $r_{\mathbf{\xi}}$ has been
made so far.
It can be parametrised with any function and hence also
describe unambiguously extreme, non-Gaussian outliers of the response which are in
in particular responsible for QCD events with large \met.
However in the following, a Gaussian function is assumed for the time
being in order to develop the method and measure the main bulk of the
response.
The Gaussian have a fixed mean value of one, \ie the
correct jet energy scale is assumed, and be dependent on the free width
parameter \mbox{$\mathcal{\xi} = \sigma$}.
In that case a different choice of coordinates than $\pti{1}$ and
$\pti{2}$ is preferential.
With
\begin{align}
  \label{eq:ResFit:TransformedCoordinates}
  \begin{split}
    \ptave     &  = \frac{1}{2}\left(\pti{1}+\pti{2}\right) \\
    \Delta\pt  &  = \frac{1}{2}\left(\pti{1}-\pti{2}\right)
  \end{split}
\end{align} 
the dijet pdf~\qeq{eq:ResFit:DijetPdf} becomes
\begin{equation}
  \label{eq:ResFit:DijetPdfTransformed}
   g_{\sigma'}\left(\ptave,\Delta\pt\right) \propto
   \e^{-\frac{1}{2}\left(\frac{\Delta\pt}{\sigma'}\right)^{2}}\cdot
   \int\dif{\pttrue}\;f\left(\pttrue\right)\cdot
   \e^{-\frac{1}{2}\left(\frac{\ptave - \pttrue}{\sigma'}\right)^{2}}
   \; ,
\end{equation}
where \mbox{$\sigma' = \sigma/\sqrt{2}$}.
It is evident from~\qeq{eq:ResFit:DijetPdfTransformed} that the dijet
pdf is in fact equivalent to a pdf of the \pt imbalance, \ie the dijet
asymmetry, with an additional assumption for the particle level jet
\pt spectrum.
This will be further discussed and exploited in \qsec{sec:ResFit:DataDriven}.
