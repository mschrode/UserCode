% $Id: Note_ResFit_Technique.tex,v 1.3 2010/05/31 13:10:51 mschrode Exp $


\section{Description of the Method}\label{sec:ResFit:Method}

\subsection{Definition of the dijet likelihood}\label{sec:ResFit:Method:Likelihood}

The underlying assumptions of the presented method are
\begin{itemize}
\item an ideal dijet event configuration with exactly two jets that
  are balanced in true transverse momentum, \mbox{$\pttrue\equiv\ptparticle$};
\item measured jet transverse momenta,
  \mbox{$\ptmeasi{i}\equiv\ptcaloi{i}$}, $i\in[1,2]$, result from uncorrelated
  calorimeter measurements.
\end{itemize}
Based on these, the pdf of a measured dijet event configuration is defined as
\begin{equation}
  \label{eq:resFit:DijetPdf}
  g_{\mathbf{\xi}}\left(\ptmeasi{1},\ptmeasi{2}\right) \propto \int^{\infty}_{0}\dif{\pttrue}\,f\left(\pttrue\right)
  \cdot r_{\mathbf{\xi}}\left(\ptmeasi{1}|\pttrue\right)
  \cdot r_{\mathbf{\xi}}\left(\ptmeasi{2}|\pttrue\right).
\end{equation}
Here,
\begin{itemize}
\item $f(\pttrue)$ denotes the pdf of the true jet \pt
  i.e. the particle level differential dijet cross section.
  In the following it is assumed to be known
  and taken from the Monte Carlo simulation
  (comp. e.g. Fig.~\ref{fig:resFit:qcd:dijetspectrum:subA}).
  Uncertainties on the simulated cross section have to be propagated to
  the measured resolution and added as a systematic uncertainty of the
  method, which is in fact small as shown in Section~\ref{};
\item $r_{\mathbf{\xi}}(\ptmeasi{i}|\pttrue)$ denotes the pdf of the measured \pt
  of the $i$-th jet.
  It is parametrised with a suitable function depending on the set of
  parameters $\mathbf{\xi}$, e.g. a Gauss or Crystal Ball function.
\end{itemize}
It is important to note that
$g_{\mathbf{\xi}}(\ptmeasi{1},\ptmeasi{2})$ does not depend on \pttrue
as this is the integration variable.

If $f(\pttrue)$ and $r_{\mathbf{\xi}}(\ptmeasi{i}|\pttrue)$ are properly normalised in the considered interval \mbox{$\pttruehigh{min} < \pttrue < \pttruehigh{max}$} such that
\begin{eqnarray*}
  1 & = & \int^{\pttruehigh{max}}_{\pttruehigh{max}}\dif{\pttrue}\,f\left(\pttrue\right) \\
  1 & = & \int^{\infty}_{0}\dif{\ptmeasi{i}}\,r_{\mathbf{\xi}}\left(\ptmeasi{i}|\pttrue\right)
\end{eqnarray*}
then also the dijet pdf is properly normalised to
\begin{equation*}
  1 = \int^{\infty}_{0}\dif{\ptmeasi{1}}\,\int^{\infty}_{0}\dif{\ptmeasi{2}}\, g_{\mathbf{\xi}}\left(\ptmeasi{1},\ptmeasi{2}\right),
\end{equation*}
provided the integration order of $\ptmeasi{i}$ and $\pttrue$ can be interchanged.

For a sample of $N$ dijet events, a likelihood is defined as
\begin{equation}
  \label{eq:ResFit:Likelihood}
  \mathcal{L}\left(\mathbf{\xi}\right) = \prod^{N}_{k=1} g_{\mathbf{\xi},k}\left(\ptmeasi{1},\ptmeasi{2}\right).
\end{equation}
Maximisation of $\mathcal{L}(\mathbf{\xi})$ then results in an
estimate of the optimal values of $\mathbf{\xi}$ for the chosen
function $r_{\mathbf{\xi}}(\ptmeasi{i}|\pttrue)$.
Note that $r_{\mathbf{\xi}}(\ptmeasi{i}|\pttrue)$ is defined as the
pdf of the measured jet \pt.
The pdf of the jet \pt response \mbox{$R_{i}\equiv\ptmeasi{i} / \pttrue$} is obtained by parameter transformation
\begin{equation*}
  r_{\mathbf{\xi}}\left(R_{i}|\pttrue\right) =
  r_{\mathbf{\xi}}\left(\ptmeasi{i}\left(R_{i}\right)|\pttrue\right)\cdot\left|\frac{\dif{\ptmeasi{i}}}{\dif{R_{i}}}\right|.
\end{equation*}



\subsection{Description of selection biases in a data driven event selection}\label{sec:ResFit:Method:Biases}

\textit{To be copied from RA2 note}
