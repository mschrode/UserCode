\section{Samples and Event Selection}\label{sec:ResFit:EvtSel}

A sample of QCD events corresponding to an integrated luminosity of $2.2\pbinv$ is
selected from the data set \texttt{JetMET\_Run2010A-PromptReco-v4}, which has been reconstructed with the \texttt{3\_6\_1\_patch7} version of the \texttt{CMSSW} software.
The data has been collected by triggering on the minimum average uncorrected transverse momentum, \mbox{$\ptave = (\pti{1} + \pti{2})/2$}, of the leading two jets in each event (\texttt{HLT\_DiJetAve15U}, \texttt{HLT\_DiJetAve50U}).
The different trigger thresholds are given in \qtab{tab:ResFit:PtAveBins}, for a listing of the trigger turn-on compare \eg~\cite{bib:cmspas:trigger}.
Events originating from beam-halo and other beam-background processes are rejected~\cite{bib:cmspas:monster} and a good reconstructed primary vertex is required~\cite{bib:cmspas:vertex}.

\begin{table}[ht]
  \caption{\ptave bins and applied trigger thresholds.}
  \centering
  \begin{tabular}[ht]{ccc}
    \toprule
    Bin & \ptave range (\gevnospace) & Trigger threshold \\
    &  & (uncorrected \ptave (\gevnospace)) \\
    \midrule
    1 & 60 -- 80 & 15 \\
    2 & 80 -- 100 & 80 \\
    3 & 100 -- 120 & 80 \\
    4 & 120 -- 140 & 80 \\
    5 & 140 -- 170 & 80 \\
    6 & 170 -- 200 & 80 \\
    7 & 200 -- 250 & 80 \\
    \bottomrule
  \end{tabular}
  \label{tab:ResFit:PtAveBins}
\end{table}

In order to estimate the particle level jet \pt spectrum and for closure tests, also simulated QCD multijet events are used.
They have been generated in bins of \pthat with PYTHIA6~\cite{bib:pythia} and processed through the full GEANT4~\cite{bib:geant} based CMS detector simulation (\texttt{/QCDDiJet\_PtXXtoXX/Spring10-START3X\_V26\_S09-v1/GEN-SIM-RECO}).

Jets are reconstructed from calorimeter towers using the
anti-$k_{T}$ jet clustering algorithm~\cite{bib:akj} with parameter size \mbox{$d=0.5$}.
Jet energy corrections~\cite{bib:cmspas:jec} are applied to remove the 
$\eta$ and \pt dependence of the jet energy scale (Spring10 \texttt{L2L3} corrections) and quality criteria rejecting jets clustered from detector noise~\cite{bib:cmspas:jetid} are applied to the
two leading jets (loose Jet-ID).
In order to select dijet events, the two leading jets are required to be
back-to-back in the transverse plane, \mbox{$|\Delta\phi_{i}| > 2.7$} for \mbox{$i=1,2$}.
Additionally they have to lie within the same pseudorapidity range \mbox{$|\eta_{i}| < 1.3$}.
This accounts for the dependence of the jet \pt response on $\eta$ due to the fact that the intrinsic calorimeter resolution is energy rather than \pt dependent as well as the varying calorimeter resolution in different pseudorapidity regions.

As discussed in \qsec{sec:ResFit:DataDriven:AddJets:Extrapolation}, events are further selected in bins of \ptave (\qtab{tab:ResFit:PtAveBins}) and events with additional jet activity are rejected in course of the analysis.
In case of the simulated samples, only events from a limited number of \pthat bins contribute to a given \ptave bin.
They are weighted in such a way that the realistic QCD \pt spectrum is preserved and the event weight in the lowest of these \pthat bins is one, thus achieving the maximally possible statistical precision.
